\documentclass[12pt, a4paper]{article}
\usepackage{blindtext, titlesec, amsthm, thmtools, amsmath, amsfonts, scalerel, amssymb, graphicx, titlesec, xcolor, multicol, bm}
\usepackage{mathtools}
\usepackage[utf8]{inputenc}
% \hypersetup{colorlinks,linkcolor={red!40!black},citecolor={blue!50!black},urlcolor={blue!80!black}}
\newtheorem{theorem}{Theorema}[subsection]
\newtheorem{lemma}[theorem]{Lemma}
\newtheorem{lem}[theorem]{Lemma}
\newtheorem{corollary}[theorem]{Corollarium}
\newtheorem{hypothesis}{Coniectura}
\theoremstyle{definition}
\newtheorem{definition}{Definitio}[section]
\theoremstyle{remark}
\newtheorem{remark}{Observatio}[section]
\newtheorem{example}{Exampli Gratia}[section]
\newcommand{\bb}[1]{\mathbb{#1}}
\renewcommand\qedsymbol{Q.E.D.}
\title{Some Properties of Simple Hyperplane Arrangements}
\author{Harry Han}
\date{\today}
\renewcommand{\emph}[1]{\textit{\textbf{#1}}}
\newcommand{\curly}{\mathrel{\leadsto}}

\begin{document}
\maketitle
%\tableofcontents
\section{Hyperplane Arrangement in $\mathbb{C}^n$}

\subsection{Definitions of Hyperplanes}

We can consider $\bb{C}^n$ as a $n$ dimensional vector space over $\bb{C}$. 
A hyperplane in $\bb{C}^n$ is the collection of points that satisfy the equation: $c_0 + c_1x_1 + c_2x_2 + c_3x_3 + \cdots c_nx_n = 0$, where $c_i, 0 \leq i \leq n, $ are not all zero. The equivalent definition is that a hyperplane is the algebraic variety $V(c_1x_1 + c_2x_2 + c_3x_3 + \cdots c_nx_n + c_0)$. 

The hyperplane can be represented as a $1\times (n+1)$ matrix recording the coefficients of the equation.

$$
P_m \coloneq 
\begin{bmatrix}
	c_1 & c_2 & c_3 & \cdots & c_0
\end{bmatrix}
$$

we can define a bijective and and continuous  mapping $\phi: P \rightarrow  \bb{C}^{n-1}$ thus (also assuming $c_i, 0 \leq i \leq n, $ are not all zero): 

For any point $p \in \bb{C}^n$, $p$ lies on the hyperplane represented by $P_m$ if and only if $p$ can be written in the form, for certain $a_1, a_2, \cdots, a_n$,

$$ p \coloneq (a_1, a_2, \cdots, a_{i-1}, \frac{c_0 + c_1a_1 + \cdots + c_na_n}{-c_i}, a_{i+1}, \cdots, a_n)$$.

Thus, regarding $\mathbb{C}^n$ as a metric space with the usual metric, we can define homeomorphism
$\phi: p \curly (a_1, a_2, \cdots, a_{i-1}, a_{i+1}, a_n)$.

This means that a hyperplane in $\bb{C}^n$ is \emph{homeomorphic} to $\bb{C}^{n-1}$.

\subsection{Hyperplane Arrangements}

A collections of hyperplanes are called \emph{hyperplane arrangements}, which can also be represented by matrices. For clarity, we represent the hyperplane arrangement itself by mathematic caligraphy letter like $\mathcal{M}$, and its matrix by subscript $mat$.

For example:
$$
\mathcal{M}_{mat} = 
\begin{bmatrix}
	c_{1,1} & c_{1,2} & c_{1,3} & \cdots & c_{1,n} & c_{1,0} \\
	c_{2,1} & c_{2,2} & c_{2,3} & \cdots & c_{2,n} & c_{2,0}
\end{bmatrix}
$$
represents a hyperplane arrangement $\mathcal{M}$ with two hyperplanes $V(c_{1,1}x_1 + c_{1,2}x_2 + \cdots + c_{1,n}x_n + c_{1,0})$ and $V(c_{2,1}x_1 + c_{2,2}x_2 + \cdots + c_{2,n}x_n + c_{2,0})$.

$\mathcal{M}_{mat}$ can be divided into coefficient matrix $\mathcal{M}_{coef}$  and and constant matrix $\mathcal{M}_{con}$:

$$
\mathcal{M}_{coef} =
\begin{bmatrix}
	c_{1,1} & c_{1,2} & c_{1,3} & \cdots & c_{1,n} \\
	c_{2,1} & c_{2,2} & c_{2,3} & \cdots & c_{2,n}
\end{bmatrix}
,
\mathcal{M}_{con} =
\begin{bmatrix}
	c_{1,0} \\
	c_{2,0}
\end{bmatrix}
$$

With these notations set, we are ready to investigate more on hyperplane arrangements.

\subsubsection{Arrangement with Two Hyperplanes}

Let $\mathcal{M}$ be an arrangement with two hyperplanes, $P$ and $Q$. 
Let $P = V(c_{1,1}x_1 + c_{1,2}x_2 + \cdots + c_{1,n}x_n + c_{1,0})$ and $Q = V(c_{2,1}x_1 + c_{2,2}x_2 + \cdots + c_{2,n}x_n + c_{2,0})$.

The corresponding matrix for $\mathcal{M}$ is
$$
\mathcal{M}_{mat} = 
\begin{bmatrix}
	c_{1,1} & c_{1,2} & c_{1,3} & \cdots & c_{1,n} & c_{1,0} \\
	c_{2,1} & c_{2,2} & c_{2,3} & \cdots & c_{2,n} & c_{2,0}
\end{bmatrix}
$$

Let us investigate if they are the same hyperplane, or, if they are disctinct, whether they intersect. 
Indeed, all we need to do is to solve this system of linear equation:
\begin{equation}\label{eq:2equation}
	\mathcal{M}_{coef} \bm{x} = - \mathcal{M}_{con}
\end{equation}

$P$ and $Q$ will represent the same hyperplane if and only if 

$$
[c_{1,1}, c_{1,2}, \cdots c_{1,0}] = k [c_{2,1}, c_{2,2}, \cdots, c_{2, 0}]
$$

for some constant $k$. 
In such case $\mathcal{M}_{mat}$ is dimension 1, so $P$ and $Q$ are distinct hyperplanes precisely when $\mathcal{M}$ will be full ranked. 

Provided $P, Q$ are distinct, they may or may not intersect. 
If they do not intersect (that is, $P \cap Q = \varnothing$), they are \emph{parallel}.

$P$ and $Q$ are parallel precisely when $\ref{eq:2equation}$ has no solution; this means the second row of the reduced row echelon form of $\mathcal{M}$ must be $[0, 0, \cdots, 1]$. In such case $\mathcal{M}_{coef}$ is dimension 1.

If $\mathcal{M}_{coef}$ is full ranked, $\mathcal{M}_{cat}$ will also be full ranked, thus $P$ and $Q$ are distinct and will intersect. 
Gaussian Elimination tells us that $P \cap U$ will be codimension 2. Arrangments like this are \emph{simple}.

We can extend our rational for arrangements with more than two hyperplanes.

\subsubsection{More than Two Hyperplanes}

Let $\mathcal{M}$ be an arrangement with $m$ hyperplanes in $\bb{C}^n$. 

The $m$ hyperplanes are all distinct precisely when every $2\times (n+1)$ submatrix of $\mathcal{M}_{mat}$ is full ranked. 
Note this condition is different from $\mathcal{M}_{mat}$ is full ranked.

We define $\mathcal{M}$ to be a simple arrangement if the intersection of any $k$ hyperplanes of $\mathcal{M}$, if exists, is codimension $k$.

The following theorem determines the necessary and sufficient condition for a hyperplane arrangement to be simple:

\begin{theorem}[Simple Arrangement]
	\label{th:simple_arrangement}
	A hyperplane arrangement, $\mathcal{M}$, in $\bb{C}^n$ is simple if and only if any $k \times (n+1) $ submatrix of $\mathcal{M}$ is full ranked or contains a row of $[0, 0, \cdots, 0, 1]$ in reduced row echelon form. 
	A $j \times  k$ matrix is full ranked if its rank is $\max(j,k)$.
\end{theorem}

\begin{proof}
	\textbf{Backward Direction}: if any $k\times (n+1)$ submatrix of $\mathcal{M}$ contains a row of $[0, 0, \cdots, 0, 1]$ in reduced row echelon form, it surely has no solution. 

	If it is full ranked, 
	let $\mathcal{L}$ be any $k \times (n+1)$ submatrix which represents a subset of hyperplanes in the arrangement.

	There are two subcases to consider: $\mathcal{L}_{coef}$  is full ranked, or $\mathcal{L}$ contains a row of $[0,0,\cdots , 0 ,1]$. 
	In former case the intersection of the hyperplane can be paramatrised after Guassian Elimination, and must be codimension $k$. In the latter case their will be no solution.

	\textbf{Forward Direction}: Instead of proving directly the forward direction, let us prove its contrapositive. 

	Assuming one of the $k \times (n+1)$ submatrix, $\mathcal{L}$, is not full ranked, nor contains a row of $[0, 0, \cdots, 0, 1]$ in reduced row echelon form.

	Let $\mathcal{L}_{reduced}$ be the reduced row echelon from of $\mathcal{L}$.
		 $\mathcal{L}_{reduced}$ must be of this form:
			$$
			\begin{bmatrix}
				1 & a & b & \cdots &  &  c_1\\
				0 & 1 & b'' & \cdots &  &  c_2\\
				\cdots \\
				0 & 0 & 0 & \cdots & 1 &  c_{m-1}\\
				0 & 0 & 0 & \cdots & 0 & 0 \\
			\end{bmatrix}
			$$
	Clearly the intersection is not codimension $k$.
\end{proof}

Here is a corollary:

\begin{corollary}
	\label{cor:intersection of m hyperplanes}
	To find the number of intersection of $m$ hyerplanes in simple arragement $\mathcal{M}$, we just need to count the number of full ranked $m \times (n+1)$ submatrices of $\mathcal{M}_{mat}$. 
\end{corollary}

\section{Hyperplane Arrangement in $\mathbb{F}_p^n$}

Our theory in the previous section can be extended for any vector space. 
In this section let us consider hyperplane arrangements in $\mathbb{F}_p^n$.


\subsection{Definitions and special Properties of $\mathbb{F}_p^n$}

A hyperplane in $\mathbb{F}_p^n$ is a $n-1$ dimension affine space represented as the algebraic variety $V(c_0 + c_1x_1 + c_2x_2 + c_3x_3 \cdots + c_nx_n)$, where at least one of the $c_i, 0 \leq i \leq n,$ is non-zero. 
All hyperplanes of $\mathbb{F}_p^n$ are $n-1$ dimension affine space and are isomorphic to $\mathbb{F}_p^{n-1}$. 
In a simple arrangement, the intersection of $k$ hyperplanes, if exists, is an $n-k$ dimension affine space.

As a finite field, we can count number of points in any affine subspace of $\mathbb{F}_p^n$. 

The following observation is important:

\begin{theorem}[Affine Spaces in $\mathbb{F}_p^n$]
	\label{th:f_p affine space}
	Any $m$ dimensional affine subspace of $\mathbb{F}_p^n$ defined as $V(c_0 + c_1x_1+c_2x_2 + \cdots + c_nx_n; d_0 + d_1x_1 + \cdots d_nx_n; \cdots)$ are isomorphic to $\mathbb{F}_p^m$. In particular, they have $p^m$ element.
\end{theorem}

\begin{proof}
	This is an elementary result that any $m$ dimensional affine subspace of $\mathbb{F}_p^n$ is isomorphic to $m$ dimensional vector subspace. Let us now count how many distinct points there are.
	A $m$ dimension vector subspaces has $m$ basis, $b_1, b_2, \cdots, b_m$, and an element belongs to it precisely when it can be written as
		$ c_1b_1 + c_2b_2, \cdots + c_m b_m $,
	for $c_i \in \mathbb{F}_p, 1\leq i \leq m $.

	For any different choice of $c_i s$, the sum will be different. (As $b_i$ are basis) Each $c_i$ has $p$ options to choose from, so there are total of $p^m$ number of elements.
\end{proof}

It is natural to ask how many points of intersections there are in a hyperplane arrangement. 
Specifically, in an arrangement with $m$ hyperplanes we are interested in how many points of intersections are strictly on $0, 1, 2, \cdots m$ hyperplanes. 
By \emph{strictly intersection}, we mean a point belongs to the intersection of $k$ hyperplanes but not on $k+1$ hyperplanes. 

\subsection{Counting, One, Two, Three}

Provided a hyperplane arrangement $\mathcal{M}$ with $m$ hyperplanes. 
Let us first count how many codimension $k$ intersection there are in a simple hyperplane arrangement.

\begin{theorem}[Counting Codimension $k$ Intersections]
	The number of codimension $k$ intersection in the simple hyperplane arrangement $\mathcal{M}$ of $m$ hyperplanes is the number of fully ranked $k \times n$ submatrices of $\mathcal{M}_{coef}$.
\end{theorem}

\begin{proof}
	Since the arrangement is simple, any codimension $k$ intersection must be the intersection of $k$ hyperplanes. 
	Simple arrangement does not guarentee the intersection of any $k$ hyperplanes exists. As shown in theorem \ref{th:simple_arrangement}, the sufficient condition of $k$ hyperplanes to intersect is that the corresponding $M_{coef}$ is fully ranked. 
	
	Thus we only need to count the fully ranked $k \times n$ submatrices.
\end{proof}

This is how to count the number of strictly intersection points.

\begin{theorem}[The Counting Theorem]
	Assuming $\Sigma$ is a simple hyperplane arrangement in $\mathbb{F}_p^n$ consists of $m$ hyperplane represented by the matrix $\mathcal{M}$. 

	Let $f_{l}$ denote the number of full ranked $l \times n
	$ submatrices of $\mathcal{M}_{coef}$.

	Since the arrangement is simple, no points are in the intersection of more than $n$ hyperplanes. 

	The number of the points that are in the intersection of $n$ hyperplane equal to the number of non-singular $n \times n$ submatrices of $\mathcal{M}_{coef}$. Let this number be $\mathfrak{N}_n$. Since there are no points in the intersection of more then $n$ hyperplanes, $\mathfrak{N}_n$ is the number of points strictly intersected by $n$ hyperplanes. 

	The number of the points strictly intersected by $n-2$ hyperplanes (denoted by $\mathfrak{N}_{n-1}$) is number of full ranked $(n-1) \times n$ sub-matrices of $\mathcal{M}_{coef}$ times $p$ minus  $n\mathfrak{N}_{n}$. That is, if there are $f_{n-1}$ full ranked $(n-1) \times n$ submatrices, $\mathfrak{N}_{n-1} = p \cdot f_{n-1} - n \cdot C(n, n-1) \mathfrak{N}_n$.

	The formulae goes on.

	$\mathfrak{N}_{n-2} = f_{n-2} \cdot p^2 -  \cdot C(n-1, n-2) \mathfrak{N}_{n-1} - \cdot C(n, n-2) \mathfrak{N}_{n}$.

	In general, 
	\begin{equation}
		\mathfrak{N}_{k} = f_{k} \cdot p^{n-k} - \sum_{i = k+1}^{n} C(i, k) \mathfrak{N}_{i}
	\end{equation}
\end{theorem}

\begin{proof}
	The rationale is straight forward. 
	We first count how many strictly intersection of $m$ hyperplanes there are by \ref{cor:intersection of m hyperplanes}. 
	By $\ref{th:f_p affine space}$ we know the intersection will be isomorphic to codimension $m$ sub vector space, and $p^m$ points will lay on it. In the end just remember to substract points that are already counted.
\end{proof}

% \section{Unimodular Matrix}
%
% A unimodular matrix is a matrix such that all of its square submatrix has determinant 1 or -1. In $\mathbb{F}_p^n$, we are to investigate hyperplane arrangements which can be represented as a unimodular matrix $\mathcal{M}$. 
%
% $\mathcal{M}$ being a unimodular matrix does not garantees that the arrangement is simple. If this additional condition is imposed, we can have this theorem: 
%
% \begin{theorem}[Number of 0 Dimension Intersection]
% 	Let $\mathcal{M}$ represents a simple hyperplane arrangement respect to $\bb{R}$ with only integer entries.
%
% 	Assume further that the matrix is unimodular respect to $\bb{R}$, it will also be unimodular for any prime $p$ in $\mathbb{F}_p$.
%
% 	Then the number of $0$ dimension hyperplane intersections stay unchanged for all prime $p$ regarded in $\mathbb{F}_p^n$.
% \end{theorem}
%

\section*{Notations and Definitions}

\begin{enumerate}
	\item $C(n, k) = \frac{n!}{k!(n-k)!}$, that is $n$ choose $k$.
	\item Coefficient matrix and constant matrix: 
	For a hyperplane arrangement represented by $m \times (n+1) $ matrix $\mathcal{M}$, 

	$$
	\mathcal{M} =
	\begin{bmatrix}
		c_{1,1} & c_{1,2} & c_{1,3} & \cdots & c_{1,n} & c_{1,0} \\
		c_{2,1} & c_{2,2} & c_{2,3} & \cdots & c_{2,n} & c_{2,0} 
			 \\ \cdots \\
		c_{m,1} & c_{m,2} & c_{m,3} & \cdots & c_{m,n} & c_{m,0}
	\end{bmatrix}
	$$
	The coefficient matrix $\mathcal{M}_{coef}$ is the $ m \times n$ matrix without the last column, and the constant matrix $\mathcal{M}_{con}$ is the last column. I.e., 
	$$
	\mathcal{M}_{coef} =
	\begin{bmatrix}
		c_{1,1} & c_{1,2} & c_{1,3} & \cdots & c_{1,n} \\
		c_{2,1} & c_{2,2} & c_{2,3} & \cdots & c_{2,n} 
			 \\ \cdots \\
		c_{n,1} & c_{n,2} & c_{n,3} & \cdots & c_{n,n}
	\end{bmatrix}
	, \mathcal{M}_{con} =
	\begin{bmatrix}
		c_{1,0} \\
		c_{2,0} \\
		\cdots \\
		c_{n,0}
	\end{bmatrix}
	$$
\end{enumerate}

\begin{definition}[Simple Arrangement]
A hyperplane arrangment $\Sigma$ is simple if any non-empty interestion of $k$ hyperplanes is codimension $k$.
We regard a single point as 0 dimension and the empty set as negative dimension.
\end{definition}

\begin{definition}[Strictly Intersection]
	If $p$ belongs to the strictly intersection of $k$ hyperplanes if it is on $k$ hyperplanes but not on $k+1$ hyperplanes.
\end{definition}

% \subsection{Central Hyperplane arrangement}
%
% Let $N=\{n_1, \dots, n_k\}$ be a collection of primitive integral vectors such that none of $n_i$'s is scalar multiple of the other. Suppose that there exists a hyperplane arrangement $\mathbb{V}$ with the following condition:
% \begin{enumerate}
%     \item a collection of normal vectors of real hyperplanes in $\mathbb{V}$ is $N$.
%     \item there exists a unique compact chamber. 
% \end{enumerate}
% We call such a collection of integral vectors $N$ \emph{central}. Then we define a hyperplane arrangement $\mathbb{V}$ to be central if the collection of integral primitive normal vectors is central. 
%
% Let $\Sigma$ be a complete unimodular fan. We call $\Sigma$ \emph{minimal} if it has no complete unimodular subfan $\Sigma'$ such that $\Sigma[1] \subset \Sigma'[1]$. 
% \begin{lem}
%     If $\Sigma$ is a minimal complete unimodular fan, then $\Sigma$ is a fan associated to a product of projective spaces. 
% \end{lem}
% \begin{proof}
%     Let $\rho_1, \dots, \rho_k$ be primitive rays of $\Sigma \subset \mathbb{R}^d$. Without loss of generality, we may assume that the first $d$ rays correspond to the standard basis. First the minimality condition implies that $d+1 \leq k \leq 2d$. If $k=d+1$, then $\rho_{d+1}=-\rho_1 + \cdots + -\rho_d$ since $\Sigma$ is unimodular and complete. If $k=d+2$, then both $-\rho_{d+1}$ and $-\rho_{d+2}$ should belong to the standard cone spanned by $\rho_1, \cdots, \rho_d$. If $-\rho_{d+1}$ belongs to the minimal cone $c_I$, then the minimal cone containing $-\rho_{d+2}$ must be $c_{I^c}$. In this case, the associated toric variety is $\mathbb{CP}^{|I|} \times \mathbb{CP}^{d-|I|}$. Similar arguement for other values of $k$. We leave it to the reader. 
% \end{proof}
%
% \begin{lem}
%     If $\Sigma$ is a minimal complete unimodular fan, then there exists a simple and unimodular hyperplane arrangement $\mathbb{V}$ which has a unique compact chamber whose normal fan is $\Sigma$. 
% \end{lem}
% \begin{proof}
%     The construction part is obvious. It is enough to show that other chambers are non-compact. In this case, the number of hyperplanes is the same as the number of ray generators. Hence, the minimiality implies that all other chamers are non-compact. 
% \end{proof}
% The converse statment is true. 
%
% \begin{lem}
%     Let $\mathbb{V}$ be a simple and unimodular hyperplane arrangement. If $\mathbb{V}$ has a unique compact chamber, then normal fan of that compact chamber is a minimal complete unimodular fan. 
% \end{lem}
%
% \begin{proof}
%     Let $\Delta$ be a unique chamber and $\Sigma=\mathrm{nf}(\Delta)$ be a normal fan. Suppose that $\Sigma$ is not minimal. Then there exists a complete unimodular subfan $\Sigma' \subset \Sigma$ with $\Sigma[1] \subset \Sigma'[1]$. Let $\rho_1, \dots, \rho_k$ be a ray generators in $\Sigma \setminus \Sigma'$. There exists a cone $C$ in $\Sigma$ which contains at least one of such $\rho_i$'s. Then the hyperplanes corresponding to rays of $C$ and such a $\rho_i$ form a bounded chamber. 
% \end{proof}
%
% Due to above lemmas, one can see that every central simple and unimodular hyperplane arrangement can be constructed by adding parallel hyperplanes to the one that has a unique compact chamber corresponding to a product of projective spaces. Therefore, for a parition $\underline{d}$, we say a central simple and unimodular hyperplane arrangement is \emph{of type $\underline{d}$} is the unique compact chamber corresponds to $\Delta^{\underline{d}}$. In particular, when $\underline{d}=d$, we call it \emph{maximal type}. 
%
% \begin{example}
%     Consider $\mathbb{R}^2$ with $5$ hyperplanes, $\{x=0\}, \{x=1\},\{y=0\},\{y=1\},\{x+y=3\}$. This can be obtained from a central simple and unimodular hyperplane arrangement whose unique compact chamber is either a square or a $2$-simplex. This is of type $\underline{d}=d=2$. 
% \end{example}
%
% \textbf{Question} Count the number of minimial unimodular fans in general, beyond the central case. 
%
%
\end{document}
