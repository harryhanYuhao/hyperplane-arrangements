\documentclass[12pt, a4paper]{article}
\usepackage{blindtext, titlesec, amsthm, thmtools, amsmath, amsfonts, scalerel, amssymb, graphicx, titlesec, xcolor, multicol }
\usepackage[utf8]{inputenc}
% \hypersetup{colorlinks,linkcolor={red!40!black},citecolor={blue!50!black},urlcolor={blue!80!black}}
\newtheorem{theorem}{Theorema}[subsection]
\newtheorem{lemma}[theorem]{Lemma}
\newtheorem{corollary}[theorem]{Corollarium}
\newtheorem{hypothesis}{Coniectura}
\theoremstyle{definition}
\newtheorem{definition}{Definitio}[section]
\theoremstyle{remark}
\newtheorem{remark}{Observatio}[section]
\newtheorem{example}{Exampli Gratia}[section]
\newcommand{\bb}[1]{\mathbb{#1}}
\renewcommand\qedsymbol{Q.E.D.}
\title{Hyperplan Arrangements}
\author{Harry Han}
\date{\today}
\renewcommand{\emph}[1]{\textit{\textbf{#1}}}
\begin{document}
\maketitle
%\tableofcontents
\section{Hyperplane Arrangement}

\subsection{Hyperplane Arrangement in $\bb{C}^n$}

We can consider $\bb{C}^n$ as a $n$ dimensional vector space over $\bb{C}$. 
Hyperplane in it are all the points that satisfy the equaiton: $c_0 + c_1x_1 + c_2x_2 + c_3x_3 + \cdots c_nx_n = 0$, where $c_i$ are not all zero.
That is, a hyperplane, $P$, is the algebraic variety $V(c_1x_1 + c_2x_2 + c_3x_3 + \cdots c_nx_n + c_0)$. 
We can write $P$ into a $1\times (n+1)$ matrix: 
$$
P = 
\begin{bmatrix}
	c_1 & c_2 & c_3 & \cdots & c_0
\end{bmatrix}
$$

We can define mapping $\phi: P \rightarrow  \bb{C}^{n-1}$ thus:
let $c_i$ be none zero, clearly the point $p \in P$ if and only if $p$ can be written as 
\\
$(a_1, a_2, \cdots, a_{i-1}, \frac{c_0 + c_1a_1 + \cdots + c_na_n}{-c_i}, a_{i+1}, \cdots, a_n)$.
We let  
\\
$\phi(p) = (a_1, a_2, \cdots, a_{i-1}, a_{i+1}, a_nF_p^n)$.
This is a homeomorphism.

Consider a hyperplane arragnement, $\Sigma$, with two different hyperplanes $P, Q$ written in matrix form:
$$
\mathcal{M} = 
\begin{bmatrix}
	c_{1,1} & c_{1,2} & c_{1,3} & \cdots & c_{1,n} & c_{1,0} \\
	c_{2,1} & c_{2,2} & c_{2,3} & \cdots & c_{2,n} & c_{2,0}
\end{bmatrix}
$$

Provided $P, Q$ are different, they will intersect if and only if $\Sigma$ is full ranked. In such case we call the arrangement \emph{simple} and the intersection is a $n-2$ dimension affine space.

Similarly we we can descripe any hyperplane arrangement in $\bb{C}^{n}$ with a $m \times ( n + 1)$ matrix. We define simple arrangement thus: 

\begin{definition}[Simple Arrangement]
A hyperplane arrangment $\Sigma$ with hyperplanes $P_1, P_2, \cdots, P_m$ is simple if the interestion of $k$ hyperplanes, if exists, is a $n-k$ dimension affine space. 
We regard a single point as 0 dimension and the empty set as negative dimension.
\end{definition}

If an arrangement is simple, any $k \times n+1 $submatrix of its matrix form shall either:  (1) full ranked, or (2) have no solution when considered as a system of linear equations.

\subsection{Hyperplane Arrangement in $F_p^n$}

Considering $F_p^n$, where $F_p$ is the field $\mathbb{N} \setminus p \bb{N}$ for a prime integer $p$. 
We can consider it as an $n$ dimension vector space over the field $F_p$. 

Our theory in the previous section can be extended to $F_p^n$:
A hyperplane in $F_p^n$ is a $n-1$ dimension affine space represented as the algebraic variety $V(c_0 + c_1x_1 + c_2x_2 + c_3x_3 \cdots + c_nx_n)$, where at least one of the $c_i$ is non-zero. All hyperplanes of $F_p^n$ are $n-1$ dimension affine space and are homeomorphic to $F_p^{n-1}$. In a simple arrangement, the intersection of $k$ hyperplanes is a $n-k$ dimension affine space.

As a finite field, we can count number of points in any affine subspace of $F_p^n$. 

\begin{theorem}[Affine Spaces in $F_p^n$]
	Any $m$ dimensional affine subspace of $F_p^n$ defined as $V(c_0 + c_1x_1+c_2x_2 + \cdots + c_nx_n; d_0 + d_1x_1 + \cdots d_nx_n; \cdots)$ are homeomorphic to $F_p^m$. In particular, they have $p^m$ element.
\end{theorem}

It is thus natural to ask how many points of intersections there are in a hyperplane arrangement. 
Specifically, in an arrangement with $m$ hyperplanes we are interested in how many points of intersections are strictly on $0, 1, 2, \cdots m$ hyperplanes. 
By \emph{strictly intersection} of $k$ hyperplanes we mean a point lays on $k$ hyperplanes but not on $k+1$ hyperplanes. 

\begin{definition}[Strictly Intersection]
	If $p$ belongs to the strictly intersection of $k$ hyperplanes if it is on $k$ hyperplanes but not on $k+1$ hyperplanes.
\end{definition}

This is how to count:

\begin{theorem}[The Counting Theorem]
	Assuming $\Sigma$ is a simple hyperplane arrangement in $F_p^n$ consists of $m$ hyperplane, and can be represented in the matrix $\mathcal{M}$. 

	Let $f_{l}$ denote the number of full ranked $l \times (n
	+ 1)$ submatrices of $\mathcal{M}$.

	Since the arrangement is simple, no points are in the intersection of more than $n$ hyperplanes. 

	The number of the points that are in the intersection of $n$ hyperplane equal to the number of non-singular $n \times (n + 1)$ submatrices of $\mathcal{M}$. Let this number be $\mathfrak{N}_n$. Since there are no points in the intersection of more then $n$ hyperplanes, $\mathfrak{N}_n$ is the number of points strictly intersected by $n$ hyperplanes. 

	The number of the points strictly intersected by $n-2$ hyperplanes (denoted by $\mathfrak{N}_{n-1}$) is number of full ranked $(n-1) \times n$ sub-matrices of $\mathcal{M}$ times $p$ minus  $n\mathfrak{N}_{n}$. That is, if there are $f_{n-1}$ full ranked $(n-1) \times n$ submatrices, $\mathfrak{N}_{n-1} = p \cdot f_{n-1} - n \cdot C(n, n-1) \mathfrak{N}_n$.

	The formulae goes on.

	$\mathfrak{N}_{n-2} = f_{n-2} \cdot p^2 -  \cdot C(n-1, n-2) \mathfrak{N}_{n-1} - \cdot C(n, n-2) \mathfrak{N}_{n}$.

	In general, 
	\begin{equation}
		\mathfrak{N}_{k} = f_{k} \cdot p^{n-k} - \sum_{i = k+1}^{n} C(i, k) \mathfrak{N}_{i}
	\end{equation}
\end{theorem}

\section{Unimodular Matrix}

A unimodular 

\section*{Notation}

\begin{enumerate}
	\item $C(n, k) = \frac{n!}{k!(n-k)!}$, that is $n$ choose $k$.
\end{enumerate}

\end{document}
